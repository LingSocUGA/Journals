\documentclass{article}
  % Packages
  \usepackage{fontspec}
    \setmainfont{Charis SIL}
  \usepackage{hyperref}
    \hypersetup{colorlinks=true, allcolors=blue}

  % Commands
  \newcommand{\tabhead}[1]{\textbf{#1}}

  % Document info
  \title{\textsc{
           University of Georgia \\
           Formulaire de consentement \\
         }
         The expression of race in Louisiana French among Creoles and Cajuns}
  \date{}

  \begin{document}
    \maketitle
    On vous demande de prendre part à une étude.
    Les renseignements dans ce formulaire vont vous aider à décider si vous voulez être dans l'étude.
    Demandez voir au chercheur ci-dessous s'il y a un renseignement qui n'est pas clair ou si vous avez besoin de renseignements supplémentaires.
    \begin{center}
      \begin{tabular}{l l}
        \tabhead{Chercheur principal:} & Joshua McNeill \\
                                       & University of Georgia \\
                                       & Linguistics Department \\
                                       & 504-264-2474 \\
                                       & \href{mailto:joshua.mcneill@uga.edu}{joshua.mcneill@uga.edu}
      \end{tabular}
    \end{center}

    \section{Objet de l'étude}
      Le chercheur effectue cette étude pour apprendre davantage de la race et l'ethnicité en Louisiane et quel rôle éventuel que la race et l'ethnicité jouent dans le parler français et créole du monde en Louisiane.

    \section{Éligibilité}
      On vous invite de prendre part à cette étude parce que vous parlez français ou créole et vous auto-identifiez comme Créole ou Cadien.

    \section{Ce qu'on vous demande de faire}
      Si vous êtes d'accord de prendre part à cette étude, vous allez:
      \begin{enumerate}
        \item Finir une interview récordée en audio avec le chercheur au sujet de votre identité, vos expériences, et vos opinions sur la race et l'ethnicité. (45-60 minutes)
        \item Finir une enquête où vous nommez des personnes dans votre réseau social et quelques renseignements sur chacun. (10-15 minutes)
        \item Finir une tâche de traduction où vous traduisez des phrases anglaises en français ou créole. (10-15 minutes)
      \end{enumerate}
      On vous demande donc de passer entre 1 heure 5 minutes et 1 heure 30 minutes avec le chercheur.

    \section{Participation volontaire}
      Vous pouvez refuser d'y prendre part ou arrêter n'importe quand sans pénalité.

    \section{Risques potentiels}
       \begin{itemize}
         \item Vous pouviez vous sentir mal à l'aise à l'idée de donner votre voix au chercheur.
         \item Vous pouviez vous sentir mal à l'aise dans la discussion de la race, l'ethnicité, ou d'autres sujets qui se présentent pendant l'interview.
       \end{itemize}

    \section{Avantages potentiels}
      \begin{itemize}
        \item Le fait de donner votre voix à la recherche sur votre langue et votre peuple pouviez vous donner satisfaction.
        \item Vous pouviez vous intéresser à la discussion de ces sujets avec le chercheur.
      \end{itemize}

    \section{Confidentiality}
      Le chercheur va faire des démarches pour protecter votre vie privée.
      L'enregistrement audio et la transcription de votre interview seront gardés sur un serveur sûr sur un site internet de stockage privé.
      Cependant, il reste toujours un risque mineur que vos renseignements seront par accident dévoilés à des personnes non liées à la recherche.
      Pour réduire ce risque:
      \begin{itemize}
        \item Les noms seront éliminés des enregistrements audio.
        \item Les noms seront anonymisés dans la transcription des enregistrements audio.
        \item Names will be anonymized in the transcript of the audio recording.
      \end{itemize}
      Réalisez voir que vos enregistrements audio et les transcriptions pourraient être partagés avec d'autres chercheurs pour des études futures.
      Dans ces cas, seules les versions censurées et anonymisées seront partagées.

    \section{Consent}
      En signant ci-dessous, moi, \rule{3cm}{0.4pt}, admets les suivants:
      \begin{itemize}
        \item Une copie des démarches de cette investigation et une description des risques, malaises, et avantages associés à ma participation m'a été donnés et discutés en détail avec moi.
        \item Je comprends que j'ai l'option de refuser de répondre à n'importe quelle question de l'interview ou de l'enquête.
        \item Je comprends que je peux couper la conversation n'importe quand.
        \item Je comprends que ma participation dans ce projet de recherche est volontaire.
      \end{itemize}

      \noindent\begin{tabular}{l l l}
                                       &                              & \\
                                       &                              & \\
        \rule{0.35\textwidth}{0.4pt}   & \rule{0.35\textwidth}{0.4pt} & \rule{0.2\textwidth}{0.4pt} \\
        Nom du chercheur               & Signature                    & Date \\
                                       &                              & \\
                                       &                              & \\
        \rule{0.35\textwidth}{0.4pt}   & \rule{0.35\textwidth}{0.4pt} & \rule{0.2\textwidth}{0.4pt} \\
        Nom du participant             & Signature                    & Date \\
      \end{tabular}

    \section{Follow-up}
      N'hésitez voir pas de demander des questions au sujet de cette recherche n'importe quand.
      Vous pouvez contacter le chercheur, Joshua McNeill, à 504-264-2474, \href{mailto:joshua.mcneill@uga.edu}{joshua.mcneill@uga.edu}.
      Si vous avez des plaintes ou des question qui concernent vos droits comme volontaire de recherche, contactez l'IRB à 706-542-3199 ou par courriel à \href{mailto:IRB@uga.edu}{IRB@uga.edu}.
  \end{document}
